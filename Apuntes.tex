%NO MODIFICAR ESTA SECCION!
\documentclass{article} % Define la clase del documento, en este caso, un artículo

\usepackage[letterpaper,margin=3cm]{geometry} % Configura el tamaño del papel y los márgenes del documento
\usepackage{graphicx} % Permite la inserción de imágenes
%\usepackage[spanish]{babel}% Activar esta configuración para informes en español, ajusta el idioma del documento
\usepackage[usenames]{color} % Permite el uso de colores definidos por nombre en el documento
\usepackage{hyperref} % Habilita enlaces y referencias dentro del documento
\hypersetup{colorlinks=true, linkcolor = black, citecolor= black} % Configura el color de los enlaces y citas
\usepackage{booktabs} % Proporciona comandos para crear tablas de alta calidad
\usepackage{natbib} % Permite el uso de citas y referencias bibliográficas con diferentes estilos
\usepackage{tikz} % Permite la creación de gráficos y diagramas vectoriales directamente en LaTeX
\usepackage{float} % Para controlar la posición de los elementos flotantes, como imágenes, con la opción [H]
\bibliographystyle{agsm} % Define el estilo de citas y bibliografía (en este caso, el estilo AGSM)
\usepackage{diagbox} % Permite crear celdas con líneas diagonales en tablas
\usepackage{listings} % Permite la inclusión y formateo de código fuente en el documento
\usepackage{xcolor} % Paquete para definir y usar colores en el documento
\usepackage{parskip} % Añade espacio entre párrafos en lugar de sangrías
\usepackage{fancyhdr} % Permite personalizar encabezados y pies de página
\usepackage{amsmath} % Proporciona una amplia variedad de entornos y comandos matemáticos

\pagestyle{fancy} % Usa el estilo fancyhdr
\fancyhf{} % Borra todos los encabezados y pies de página
\renewcommand{\headrulewidth}{0pt}
\renewcommand{\footrulewidth}{0pt} % Desactiva la línea horizontal predeterminada en el pie
\setlength{\headheight}{2cm} % Ajusta la altura del encabezado para hacer espacio para la línea
\fancyhead[L]{\raisebox{0.20cm}{\textbf{Fluid Mechanics}}} % Añade el texto en la parte izquierda del encabezado, subiéndolo ligeramente
\fancyhead[R]{\raisebox{0.1cm}{\includegraphics[width=0.25\linewidth]{LOGO_UNIVERSIDAD.jpg}}} % Añade la imagen en la parte derecha del encabezado y súbela un poco
\fancyhead[C]{\rule{\textwidth}{0.6pt}} % Añade una línea horizontal superior centrada
\fancyfoot[C]{\rule{\textwidth}{0.6pt}} % Añade una línea horizontal en el pie de página centrada
\fancyfoot[R]{\raisebox{-1.5\baselineskip}{\thepage}} % Coloca el número de página a la derecha, con suficiente espacio debajo de la línea
\geometry{top=3cm, bottom=2.5cm} % Ajusta los márgenes superior e inferior

% Definición de colores al estilo Visual Studio Code
\definecolor{codegreen}{rgb}{0.25,0.49,0.48} % Comentarios
\definecolor{codegray}{rgb}{0.5,0.5,0.5} % Números y anotaciones
\definecolor{codepurple}{rgb}{0.58,0,0.82} % Palabras clave
\definecolor{backcolour}{rgb}{0.95,0.95,0.92} % Color de fondo

% Configuración del estilo de las celdas de código
\lstset{
    backgroundcolor=\color{backcolour},   % color de fondo; necesita que el paquete color o xcolor esté cargado
    commentstyle=\color{codegreen},       % estilo de comentarios
    keywordstyle=\color{codepurple},      % estilo de palabras clave
    numberstyle=\tiny\color{codegray},    % estilo de los números de línea
    stringstyle=\color{red},              % estilo de las cadenas de texto
    basicstyle=\ttfamily\small,           % estilo del texto básico
    breakatwhitespace=false,              % ajustes de líneas sólo en espacios en blanco
    breaklines=true,                      % ajustar las líneas si son muy largas
    captionpos=b,                         % posición de la leyenda (abajo)
    keepspaces=true,                      % preserva los espacios en el texto; útil si se usa monoespaciado
    numbers=left,                         % dónde poner los números de línea
    numbersep=5pt,                        % qué tan lejos están los números de línea del código
    showspaces=false,                     % mostrar espacios con subrayados particulares; reemplaza 'showstringspaces'
    showstringspaces=false,               % subrayar los espacios dentro de las cadenas solo
    showtabs=false,                       % mostrar tabulaciones en el código con subrayados particulares
    tabsize=2,                            % tamaños de tabulación a 2 espacios
    language=TeX,                         % lenguaje del código
    morecomment=[l]\#,                    % reconocer # como inicio de comentario en Python
    frame=single,                         % agregar un marco simple alrededor del código
    rulecolor=\color{black}               % color del marco
}

\begin{document}
%----------------------------------------------------------------------------------------
%   PORTADA
%Modificar desde aqui en adelante
%----------------------------------------------------------------------------------------
\begin{titlepage}%Inicio de la carátula, solo modificar los datos necesarios
\newcommand{\HRule}{\rule{\linewidth}{0.5mm}} 
\center 
%----------------------------------------------------------------------------------------
%	ENCABEZADO
%----------------------------------------------------------------------------------------
\includegraphics[width=10cm]{LOGO_UNIVERSIDAD.jpg}\\ % Si esta plantilla se copio correctamente, va a llevar la imagen del logo de la facultad.OBS: Es necesario incluir el paquete: graphicx
\vspace{3cm}
%----------------------------------------------------------------------------------------
%	SECCION DEL TITULO
%----------------------------------------------------------------------------------------
\HRule \\[0.4cm]
{ \huge \bfseries Tittle}\\[0.4cm] % Titulo del documento
{ \huge \bfseries Fluid Mechanics}\\[0.4cm] % Titulo del documento
\HRule \\[1.5cm]
 \vspace{5cm}
%----------------------------------------------------------------------------------------
%	SECCION DEL AUTOR
%----------------------------------------------------------------------------------------
\begin{flushright}
    { \textbf{Profesors:}\\
    Patricio Moreno\\
    Sebastian Sepulveda\\
    \vspace{0.2cm}
    \textbf{Assistant:}\\
    Lukas Wolff\\
    \vspace{0.2cm}
    \textbf{Author:}\\
    Your name\\
}
\end{flushright}
\vspace{1cm}
%----------------------------------------------------------------------------------------
%	SECCION DE LA FECHA
%----------------------------------------------------------------------------------------
{\large \textbf{\today}}\\[2cm] % El comando \today coloca la fecha del dia, y esto se actualiza con cada compilacion, en caso de querer tener una fecha estatica, reemplazar el \today por la fecha deseada
\end{titlepage}

%----------------------------------------------------------------------------------------
\newpage

\section{Terminal}

\begin{itemize}
    \item \texttt{ls}: Comando para listar los archivos y directorios en el directorio actual.
    \item \texttt{cd directorio}: Comando para cambiar de directorio.
    \item \texttt{code .}: Comando para abrir Visual Studio Code en el directorio actual.
    \item \texttt{touch archivo.extension}: Comando para crear un archivo.
\end{itemize}

\section{GitHub}

\texttt{git clone url}: Comando para clonar un repositorio de GitHub.

\subsection{Subir Cambios}

Para subir cambios, es necesario guardarlos en un área de preparación.

\begin{itemize}
    \item \texttt{git add .}: Comando para agregar todos los archivos al área de preparación. Sube todos los cambios al área de preparación.
    \item \texttt{git add archivo}: Comando para agregar un archivo específico al área de preparación. Sube solo ese archivo al área de preparación.
\end{itemize}

Después, hay que asignar un comentario al cambio.

\begin{itemize}
    \item \texttt{git commit -m "comentario"}: Comando para confirmar los cambios y asignar un comentario.
\end{itemize}

Finalmente, se suben los cambios al repositorio remoto.

\begin{itemize}
    \item \texttt{git push -u origin main}: IMPORTANTE: solo se usa \texttt{-u} la primera vez en la rama \texttt{main}, de esta forma queda como la rama principal.
    \item \texttt{git push origin main}: Comando para subir los cambios a \texttt{main}.
\end{itemize}

\subsection{Descargar Cambios}

Simplemente ejecuta:

\begin{itemize}
    \item \texttt{git pull}: Comando que descarga cambios en la rama actual.
\end{itemize}

\textbf{OJO}: Recordar que antes de empezar a trabajar, SIEMPRE hay que ejecutar \texttt{git pull}.

\subsection{Git Ignore}

Git Ignore es un archivo que permite ignorar archivos o directorios específicos. Para crearlo, se debe ejecutar:

\begin{itemize}
    \item \texttt{touch .gitignore}
\end{itemize}

Todos los archivos que esten dentro no seran subidos al git hub.

\subsection{Ramas en GitHub}

Para crear una rama:

\begin{itemize}
    \item \texttt{git checkout -b Rama\_Lukas}
\end{itemize}

Para ver en qué rama estoy:

\begin{itemize}
    \item \texttt{git branch}
\end{itemize}

Luego, para subir cambios:

\begin{itemize}
    \item \texttt{git push nombre\_rama}
\end{itemize}

Para cambiar de rama:

\begin{itemize}
    \item \texttt{git switch nombre\_rama}
\end{itemize}

\subsubsection{Mergear una rama con \texttt{main}}

Primero, hay que estar en \texttt{main}:

\begin{itemize}
    \item \texttt{git switch main}
\end{itemize}

Ahora, traigo la rama a mergear:

\begin{itemize}
    \item \texttt{git merge nombre\_rama}
\end{itemize}

Finalmente, resuelvo posibles conflictos y subo los cambios al repositorio remoto:

\begin{itemize}
    \item \texttt{git add .}
    \item \texttt{git commit -m "mensaje"}
    \item \texttt{git push origin main}
\end{itemize}

\end{document}
